% The contents of this file is 
% Copyright (c) 2009- Charles R. Severance, All Righs Reserved

\chapter{Prefazione}

\section*{Python for Informatics : il remix di un Open Book}

E' abbastanza naturale per gli accademici, che si sentono dire continuamente {``}pubblica o muori'', voler sempre creare da zero qualcosa che sia una loro nuova creazione. Questo libro \`{e} un esperimento: non partire da zero, ma invece {``}remixare'' il libro dal titolo
\emph{Think Python: Come pensare come uno scienziato informatico}
scritto da Allen B. Downey, Jeff Elkner, ed altri.

Nel dicembre del 2009, mi stavo preparando a tenere il corso
{\bf SI502 - Networked Programming} presso l'Universit\`{a} del Michigan per il quinto semestre consecutivo e ho deciso che era tempo di scrivere un libro di testo su Python che si concentrasse sull'esplorazione dei dati invece che sulla comprensione di algoritmi ed astrazioni. Il mio obiettivo nel SI502 \`{e} quello di insegnare le tecniche fondamentali di analisi dei dati utilizzando Python. Pochi dei miei studenti avevano in progetto di diventare programmatori professionisti. Altri invece pianificavano di diventare bibliotecari, manager, avvocati, biologi, economisti, o altro, e desideravano imparare ad utilizzare abilmente le tecnologie nei rispettivi campi professionali.

Non ho mai trovato per il mio corso un libro su Python che fosse perfettamente orientato alla gestione dei dati, cos\`{i} ho deciso di scriverlo. Fortunatamente, tre settimane prima che iniziassi a lavorarci approfittando delle vacanze, in una riunione di facolt\`{a} il Dr. Atul Prakash mi ha mostrato il libro  \emph{Think Python} che lui stesso aveva usato per il suo corso. Si trattava di un testo di Informatica ben scritto, focalizzato su dirette e brevi spiegazioni dirette che facilitano l'apprendimento.

La struttura complessiva libro \`{e} stata modificata per arrivare a gestire i problemi di analisi dei dati il pi\`{u} rapidamente possibile e per fornire una serie di esercizi ed esempi ed esercizi in merito fin dall'inizio.

I capitoli 2-10 sono simili \emph{Think Python} ma sono state fatte importanti modifiche. Gli esempi e gli esercizi orientati alla gestione di numeri sono stati sostituiti con esercitazioni orientati ai dati. Gli argomenti sono presentati in un ordine tale da fornire soluzioni di analisi dei dati via via sempre pi\`{u} sofisticate. Alcuni argomenti come {\tt try} e
{\tt except} sono stati anticipati e presentati come parte del capitolo sull'esecuzione condizionale. Piuttosto che essere trattate gi\`{a} dall'inizio in maniera astratta, le funzioni sono state trattate pi\`{u} superficialmente sino a che non sono diventate necessarie per gestire la complessit\`{a} dei programmi. Quasi tutte le funzioni definite dall'utente sono state rimosse dai codici di esempio ed esercitazione al di fuori del Capitolo 4.
La parola {``} ricorsivo''\footnote{	 ad eccezione, ovviamente, di questa riga.}  non viene mai utilizzata in nessuna parte del libro.

Nei capitoli 1 e 11-16, tutto il materiale \`{e} nuovo di zecca, \`{e} focalizzato sull'uso di Python in applicazioni nel mondo reale e fornisce semplici esempi per l'analisi dei dati, comprendendo regolari espressioni per la ricerca e l'analisi, automatizzazione delle attivit\`{a} sul computer, il recupero dei dati attraverso la rete, prelievo di dati da pagine web, utilizzo di servizi web, analisi di dati in formato XML e JSON, e la creazione e l'utilizzo di database utilizzando lo Structured Query Language.
L'obiettivo finale di tutti questi cambiamenti \`{e} il passaggio da una Scienza dell'Informazione (\emph{Computer Science}) ad un'informatica il cui focus \`{e} quello di includere in un corso di primo livello solo quegli argomenti che potranno tornare utili anche a coloro che non sceglieranno di diventare programmatori professionisti.


Gli studenti che troveranno questo libro interessante e che desiderano esplorare ulteriormente l'argomento dovrebbero considerare il libro di Allen B. Downey \emph{Think Python}  Date le molte sovrapposizioni tra i due libri, gli studenti saranno in grado di acquisire rapidamente alcune ulteriori competenze nei settori addizionali sulla tecnica di programmazione tecnica e di pensiero algoritmico che sono parte di  \emph{Think Python}.
Inoltre, dato che i due libri hanno uno stile di scrittura molto simile sar\`{a} facile muoversi all'interno del libro.

\index{Creative Commons License}
\index{CC-BY-SA}
\index{BY-SA}
\small{
Come detentore del copyright su \emph{Think Python},
Allen mi ha dato il permesso di cambiare la licenza del materiale dal suo libro che viene incluso in questo libro dalla 
GNU Free Documentation License alla pi\`{u} recente di Creative Commons Attribuzione-Condividi allo stesso modo. Questo segue il generale cambiamento nelle licenze di documentazione aperta che si stanno spostando da GFDL a CC BY-SA (vedi Wikipedia). L'utilizzo della licenza CC BY-SA indica ai fruitori dell'opera che essa pu\`{o} essere utilizzata, diffusa e anche modificata liberamente, pur nel rispetto di alcune condizioni essenziali e rende ancora pi\`{u} semplice ai nuovi autori riutilizzare di questo materiale.}

Ritengo che questo libro sia un esempio del perch\'{e} i materiali aperti sono cos\`{i} importanti per il futuro della formazione, voglio ringraziare Allen B. Downey e la Cambridge University Press per la loro decisione lungimirante nel rendere il libro disponibile sotto un open Copyright. Spero che siano soddisfatti dei risultati del mio impegno e, mi auguro lo sia anche il lettore per lo sforzo collettivo di tutti noi.


Vorrei ringraziare Allen B. Downey e Lauren Cowles per il loro aiuto, la pazienza, e la guida nell'affrontare e risolvere i problemi di copyright circa questo libro.


\small{
Charles Severance\\
www.dr-chuck.com\\
Ann Arbor, MI, USA\\
September 9, 2013}

Charles Severance \`{e} Clinical Associate Professor presso l'Universit\`{a} del Michigan - School of Information.

\clearemptydoublepage

% TABLE OF CONTENTS
\begin{latexonly}

\tableofcontents

\clearemptydoublepage

\end{latexonly}

% START THE BOOK
\mainmatter

